\documentclass{beamer}
\usetheme[
  outer/progressbar=frametitle
]{metropolis}

\usepackage[portuguese]{babel}
\usepackage[utf8]{inputenc}

\hypersetup{colorlinks=true,urlcolor=blue,linkcolor=blue,citecolor=blue}

\title{Compiladores 2018.1}
\subtitle{Análise da linguagem Ruby e seu suporte a PEG}

\author[Abrev.]{Júlia Falcão, Raffael Paranhos, Wallace Baleroni}
\institute[UFF]{Universidade Federal Fluminense}

\date{21 de Março de 2018}

\begin{document}

% TITLE PAGE

\begin{frame}[plain]

\titlepage

\end{frame}

% RUBY (1 - INTRO)

\begin{frame}[fragile]{Ruby}
\textbf{Ruby} é uma linguagem de programação desenvolvida em 1995 no Japão por Yukihiro Matsumoto, e está atualmente entre as 10 linguagens de programação mais populares. Algumas de suas características são:

\begin{itemize}

\item Ruby é multiparadigma: suporta programação orientada a objetos (\textit{class-based}), funcional, imperativa e reflexiva.

\item Tem tipagem dinâmica e forte: todos os "tipos primitivos" são na verdade classes, e toda variável deve ter um, mas ele pode ser alterado dinamicamente.

\item Sua implementação padrão é escrita em C.

\item É interpretada e tem gerenciamento de memória automático.

\item Seu criador queria uma linguagem de \textit{script} mais poderosa que Perl, e mais orientada a objetos do que Python.

\end{itemize}

\end{frame}


% RUBY (2 - INSTALAÇÃO)
\begin{frame}[fragile]{Ruby}
A instalação de Ruby é bem simples, com várias maneiras diferentes disponíveis no site oficial:

\begin{itemize}
\item \textbf{Instaladores}: \href{https://github.com/rbenv/ruby-build}{ruby-build} e \href{https://github.com/postmodern/ruby-install}{ruby-install} para sistemas UNIX; \href{https://rubyinstaller.org/}{RubyInstaller} para Windows.

\item \textbf{Gerenciadores de pacotes} (em sistemas UNIX):

\begin{itemize}
\item Homebrew (macOS): \verb|$ brew install ruby|
\item apt (Ubuntu, Debian): \verb|$ sudo apt-get install ruby-full|
\item pacman (Arch Linux): \verb|$ sudo pacman -S ruby|
\end{itemize}

\item \textbf{Compilando a partir do \href{https://www.ruby-lang.org/en/downloads/}{código fonte}}:
\begin{verbatim}
$ ./configure
$ make
$ sudo make install
\end{verbatim}

\end{itemize}


\end{frame}

% RUBY (1)

\begin{frame}[fragile]{Ruby}
% [fragile]: necessário pro uso de \verbatim

Além da programação \textbf{imperativa}, onde o código diz como algo será feito, Ruby também suporta o paradigma \textbf{declarativo}, onde é dito apenas o que se quer fazer, sem especificar como.

\underline{Exemplo}: retornar subconjunto de elementos ímpares de uma lista

\begin{itemize}
\item \textbf{Imperativa:} manualmente adicionar elementos ímpares ao resultado

\begin{verbatim}
odds = []
array.each do | element |
    odds << element if element.odd?
end
\end{verbatim}

\item \textbf{Declarativa:} usar \textit{select}
\begin{verbatim}
array.select { | element | element.odd? }
\end{verbatim}

\end{itemize}

\end{frame}

% RUBY (2)

\begin{frame}{Ruby}

Ruby é uma linguagem \textbf{interpretada}, e o processo acontece em 4 fases:

\begin{enumerate}

\item \textbf{Tokenizing:} O programa é quebrado em pequenos pedaços chamados \textit{tokens}.

\item \textbf{Lexing:} Dados adicionais são acrescentados aos \textit{tokens}.

\item \textbf{Parsing:} O texto é transformado em uma árvore sintática abstrata (AST), uma estrutura de dados que representa o programa em memória.
\\*Até a versão 1.9, o código podia ser executado diretamente pela análise da AST, mas atualmente há um passo a mais.

\item \textbf{Compilação:} Finalmente, a AST é compilada para bytecode, que é então executado pela máquina virtual de Ruby.

\end{enumerate}

\end{frame}


% JRUBY

\begin{frame}{JRuby}
É uma implementação de Ruby que roda sobre a Java Virtual Machine (JVM). Isso permite o uso da linguagem Ruby normalmente mas com todas as vantagens da JVM, incluindo:

\begin{itemize}

\item \textbf{Portabilidade:} a JVM pode ser emulada em qualquer sistema  que suporte C++, trazendo muito mais portabilidade para Ruby do que a Ruby VM.

\item \textbf{Performance:} JRuby tem maior \textit{overhead} na inicialização, mas em retorno o \textit{throughput} é bem maior, o que faz valer a pena usá-la para aplicações maiores como aplicações Rails.

\item \textbf{Concorrência:} Ao contrário da implementação original, JRuby roda \textit{threads} simultaneamente, o que torna o programa mais rápido se ele tiver sido escrito corretamente para isso.

\end{itemize}


\end{frame}

\begin{frame}[fragile]{Exemplos}
\begin{itemize}
\item Hello World:
\begin{verbatim}
  puts 'Hello World!'
\end{verbatim}
\item Input:
\begin{verbatim}
  print 'Please type name >'
  name = gets.chomp
  puts "Hello #{name}."
\end{verbatim}
\end{itemize}
\end{frame}
\begin{frame}[fragile]{Insertion Sort}
\begingroup
\fontsize{10pt}{11pt}
\begin{verbatim}
def insertion_sort(array)
    final = [array[0]]
    array.delete_at(0)
    for i in array
        final_index = 0
        while final_index < final.length
            if i <= final[final_index]
                final.insert(final_index,i)
                break
            elsif final_index == final.length-1
                final.insert(final_index+1,i)
                break
            end
            final_index+=1
        end
    end
    final
end
\end{verbatim}
\endgroup
\end{frame}
\begin{frame}[fragile]{Definição de Classe}
\begingroup
\fontsize{10pt}{11pt}
\begin{verbatim}
class Customer
   @@no_of_customers = 0
   def initialize(id, name, addr)
      @cust_id = id
      @cust_name = name
      @cust_addr = addr
   end
   def display_details()
      puts "Customer id #@cust_id"
      puts "Customer name #@cust_name"
      puts "Customer address #@cust_addr"
   end
   def total_no_of_customers()
      @@no_of_customers += 1
      puts "Total number of customers: #@@no_of_customers"
   end
end
\end{verbatim}
\endgroup
\end{frame}

\begin{frame}{Suporte à construção de compiladores}

Apesar do desempenho relativamente fraco de Ruby torná-la muito mais adequada para aplicações \textit{web} do que a construção de compiladores, o único fator realmente decisivo é se a linguagem possui ou não as estruturas necessárias para construir o \textit{parser}.

Com o uso de bibliotecas externas como Treetop e Parslet, isso é perfeitamente possível, tanto que podemos encontrar na Internet projetos de compiladores escritos em Ruby para linguagens como x86 assembly.

\end{frame}


% RUBINIUS

\begin{frame}{Rubinius}
Rubinius é uma implementação alternativa criada com o objetivo de implementar nativamente o máximo possível da linguagem Ruby usando Ruby.
No início do projeto, a biblioteca principal era escrita quase inteiramente em Ruby, assim como o compilador de \textit{bytecode} e o \textit{debugger}.
No entanto, Rubinius acabou sendo a primeira implementação de Ruby baseada em C++, utilizando mais C++ do que Ruby no código. A ideia é interessante e é um projeto ambicioso, mas na prática, Rubinius ainda não cumpre a promessa de alto desempenho.

\end{frame}


\begin{frame}{Parsing Expression Grammars (PEGs)}
\begin{itemize}
\item Criado por Bryan Ford, é um formalismo que descreve um conjunto de regras para reconhecer \textit{strings} em uma linguagem.
\item É semelhante à Gramática Livre de Contexto, porém o operador de escolha na PEG seleciona a primeira correspondência enquanto na GLC é ambíguo.
\item Não existe ambiguidade, apenas uma árvore de derivação existe para cada \textit{string}
\item A biblioteca Parslet permite a construção de \textit{parsers} da forma PEG no Ruby.
\end{itemize}
\end{frame}

\begin{frame}[fragile]{Parslet}
O processo de criação de um compilador ou interpretador para uma linguagem pode ser dividido em quatro estágios:

\begin{enumerate}
\item \textit{Parsing}
\item Construção da Árvore Sintática Abstrata
\item Otimização da Árvore
\item Geração do código ou execução
\end{enumerate}

O Parslet nos auxiliará nos dois primeiros passos através das classes:
\begin{verbatim}
   Parslet::Parser
   Parslet::Transform
\end{verbatim}

\end{frame}

\begin{frame}[fragile]{Instalação e Exemplos Parslet}
A biblioteca Parslet está disponível na forma de Pacote \textit{Gem} e pode ser instalado com o comando:
\begin{verbatim}
   gem install parslet
\end{verbatim}
Criar um \textit{parser} que reconhece numeros é bem simples:
\begin{verbatim}
   class Mini < Parslet::Parser
      rule(:integer) { match('[0-9]').repeat(1) }
      root(:integer)
   end
\end{verbatim}
\end{frame}

\begin{frame}[fragile]{Mais exemplos}
Extendendo um pouco o \textit{parser} anterior, é possível criar um  que reconhece algumas expressões, por exemplo "2 + 2":
\begin{verbatim}
class Mini < Parslet::Parser
    rule(:integer)    { match('[0-9]').repeat(1)>>space? }
    rule(:space)      { match('\s').repeat(1) }
    rule(:space?)     { space.maybe }
    rule(:operator)   { match('[+]')>>space? }
    rule(:sum)        { integer>>operator>>expression }
    rule(:expression) { sum | integer }
    root :expression
end
\end{verbatim}
\end{frame}

\begin{frame}[fragile]{Erros}
O Parslet auxilia também proporcionando o melhor relatório de erros possível. A expressão "a + 2" no \textit{parser} anterior gera o seguinte erro:
\begin{verbatim}
Expected one of [SUM, INTEGER] at line 1 char 1.
|- Failed to match sequence (INTEGER OPERATOR EXPRESSION) at line 1 char 1.
|  `- Failed to match sequence ([0-9]{1, } SPACE?) at line 1 char 1.
|     `- Expected at least 1 of [0-9] at line 1 char 1.
|        `- Failed to match [0-9] at line 1 char 1.
`- Failed to match sequence ([0-9]{1, } SPACE?) at line 1 char 1.
   `- Expected at least 1 of [0-9] at line 1 char 1.
      `- Failed to match [0-9] at line 1 char 1.
\end{verbatim}
\end{frame}

\begin{frame}[fragile]{Transform}
O \textit{parser} tem como saída uma estrutura difícil de ser trabalhada, as \textit{deep nested hashes}. Por isso, o Parslet possui uma classe que transforma essa saída.

Um transformador que coloca "b" no lugar de "a" pode ser escrito da seguinte maneira:
\begin{verbatim}
 class MyTransform < Parslet::Transform
    rule('a') { 'b' }
  end
\end{verbatim}
A regra de transformaçao possui duas partes, um padrão, o qual deverá ser reconhecido e substituido (a) e um bloco que possui o que deverá ser colocado no lugar do padrão (b).
\end{frame}

% ---

\begin{frame}{Prós e contras}

Desvantagem: \textbf{desempenho}

Ruby é considerada uma linguagem lenta em comparação com outras linguagens de alto nível, e mais ainda em comparação com C, a linguagem mais popularmente usada na construção de compiladores. Isso se dá por algumas razões:

\begin{itemize}
\item É uma linguagem interpretada, e essas tendem a ser mais lentas que linguagens compiladas.
\item O \textit{garbage collector} de tempos em tempos para a execução do programa para limpar a memória alocada não utilizada.
\item Chamadas de métodos são lentas pois o código não possui um ponteiro para os métodos e sim o nome para buscá-lo na classe a qual ele pertence, ou nos ancestrais dessa classe.
\item Não usa \textit{multithreading}; o programa inteiro é executado em uma única \textit{thread}.

\end{itemize}

\end{frame}


\begin{frame}{Prós e contras}
Vantagens:

\begin{itemize}
\item \textbf{Legibilidade:} Por ser uma linguagem de alto nível, a sintaxe de Ruby é simples e inteligível, e atividades complexas podem ser executadas em poucos comandos, característica a qual muitos se referem como a "mágica" do Ruby.

\item \textbf{Velocidade de escrita:} Ruby é particularmente útil em situações onde a maior produtividade do programador é mais importante que a velocidade da execução. Muitos desenvolvedores \textit{web} consideram justo sacrificar o desempenho pela rapidez na qual escrevem código em Ruby.

\end{itemize}

\end{frame}



\end{document}
