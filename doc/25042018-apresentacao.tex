\documentclass{beamer}

\usetheme[
  outer/progressbar=frametitle
]{metropolis}

\usepackage[portuguese]{babel}
\usepackage[utf8]{inputenc}
\usepackage{fancyvrb} % for "\Verb" macro

\hypersetup{colorlinks=true,urlcolor=blue,linkcolor=blue,citecolor=blue}

\title{DeceptiCompiler}

\subtitle{Mais do que os olhos podem ver.}

\author[Abbrev.]{P1 de Compiladores (2018.1), por Raffael Paranhos, Wallace Baleroni e Júlia Falcão}

\institute[UFF]{Universidade Federal Fluminense}

\date{25 de Abril de 2018}

\begin{document}

% ---

\begin{frame}[plain]

\titlepage

\end{frame}

% ---

\begin{frame}[fragile]{Parser}
Classe \verb|OptimusParser < Parslet::Parser|, contendo regras do tipo:
\begin{verbatim}
rule(:exp) { mathexp | boolexp | integer | ident }
\end{verbatim}

Uma "expressão" pode ser uma exp. matemática, booleana, um inteiro ou um identificador (uma variável).

\begin{verbatim}
rule(:ini) { ident >> ini_op >> exp }
\end{verbatim}
Uma inicialização é composta por um identificador, o operador e uma "expressão".


A regra na qual o parser inicia o processo é declarada por:
\begin{verbatim}
root(:ex_proc)
\end{verbatim}

\end{frame}

% ---

\begin{frame}[fragile]{Parser}
\begin{itemize}
\item \textbf{Expressão  aritmética}:
\begin{verbatim}
rule(:mathexp) { (ident | integer) >> arithop
>> (mathexp | ident | integer) }
\end{verbatim}


\item \textbf{Expressão booleana}:
\begin{verbatim}
rule(:boolexp) { neg_op.maybe >> (ident | integer)
>> boolop >> exp }
\end{verbatim}

\item \textbf{Comando}:
\begin{verbatim}
rule(:cmd) { (cmd_unt >> cho_op >> cmd |
cmd_unt >> seq_op >> cmd | cmd_unt) }

rule(:cmd_unt) { ex_if | ex_while | ex_print |
ex_exit | call | ident >> ass_op >> exp }
\end{verbatim}

\end{itemize}

\end{frame}

\begin{frame}[fragile]{Parser}
Executando o parse de um módulo inteiro:
\begin{verbatim}
OptimusParser.new.rollOut("
    module Fact
    var y, x
    init y = 1 

    proc fact(x) {
        if (~ x == 0) {
            y := x * y ; y := x + 1
        } 
    }
    end
");
\end{verbatim}

\end{frame}

\begin{frame}[fragile]{Transform}
A classe \verb|Bumblebee < Parslet::Transform| recebe a saída do parser e determina como e onde cada parte do código será empilhada, inicializando a estrutura SMC.

Para fazer isso, são usados \textit{aliases} no código do \verb|OptimusParser| que identificam as partes de cada expressão/comando para que o transform as reconheça.

\begin{verbatim}
rule(:mathexp) { (ident | integer).as(:left) >> arithop
>> (mathexp | ident | integer).as(:right) }

rule(:sum_op) { str("+").as(:op) >> blank? }

\end{verbatim}

\end{frame}

\begin{frame}[fragile]{Transform}
Nessa classe \verb|Bumblebee|, são implementadas estruturas para cada tipo de expressão/comando, que avaliam os componentes e constroem o \verb|SMC| correspondente.

\begin{verbatim}
Addition = Struct.new(:left, :right) do
  def eval
    $smc.empilhaControle('add')
    $smc.empilhaControle(left.eval)
    $smc.empilhaControle(right.eval)
  end
end
\end{verbatim}
No caso de uma adição, por exemplo, empilhamos na pilha de controle a string "add", o lado esquerdo da expressão, e por fim o lado direito.

\end{frame}

\begin{frame}[fragile]{SMC}
A classe SMC é usada para representar uma instância da estrutura \verb|<S, M, C>|, que tem como atributos as pilhas de controle, de valor e a lista de memória. Nessa lista, um item é o nome do identificador e o item seguinte é seu valor atual guardado em memória.

Exemplo:
\begin{verbatim}
@memoria = ["x", 5, "y", 1]
\end{verbatim}

A classe também define as operações para manipular a estrutura, ou seja, empilhar e desempilhar itens nas pilhas, acessar a memória e escrever nela.

\end{frame}



\begin{frame}[fragile]{BPLC}
A classe \verb|BPLC| possui o método \verb|vamosRodar| que recebe uma instância de \verb|SMC| e vai desempilhando o que está nas pilhas e executando as operações devidas.

\begin{verbatim}
val = smc.topoControle()
if (val == 'assign') smc.ce()
\end{verbatim}

\textbf{smc.rb}
\begin{verbatim}
def ce()
  self.desempilhaControle()
  val = self.desempilhaValor()
  ident = self.desempilhaValor()
  self.escreveMemoria(ident, val)
end
\end{verbatim}

\end{frame}


\begin{frame}[fragile]{Execução}
Exemplo:

\verb|code = "x := 2"  #| código em IMP\newline\newline
\verb|$smc = SMC.new  #| cria instância de SMC\newline\newline
\verb|bplc = BPLC.new  #| cria instância de BPLC\newline\newline
\verb|Bumblebee.new.apply(OptimusParser.new.rollOut(code)).eval|\newline
\verb|#| roda o parser no código em IMP e passa a saída para o método que aplica a transformação e constrói o SMC\newline\newline
\verb|bplc.vamosRodar($smc)|\newline
\verb|#| chama o método de BPLC que vai executar o programa a partir do SMC

\end{frame}

\begin{frame}[fragile]{Saída}

\begin{itemize}
\item \textbf{Saída do parser}
\begin{verbatim}
{:cmd => {:ident => {:id => "x" @0},
:ass_op => ":= " @2, :val => {:int => "2" @5}}}
\end{verbatim}

\item \textbf{Situação do SMC a cada passo}
\begin{verbatim}
Controle   ["x", "ident", 2, "assign"]
Valor      [1, 1, "x"]
Memoria    ["x", 1, "y", 720]
\end{verbatim}

\item \textbf{Regras}
\begin{verbatim}
E op E
C := E
\end{verbatim}

\item \textbf{Comandos de \textit{print}}
\begin{verbatim}
Print
720
\end{verbatim}

\end{itemize}

\end{frame}

\begin{frame}[fragile]{Make}
O comando \verb|make| instala as dependências necessárias para o compilador funcionar e roda o teste do fatorial de x = 6.

Testes adicionais podem ser executados digitando \verb|make [test]|, e os testes disponíveis são \verb|add|, \verb|sub|, \verb|mul|, \verb|div| \verb|while|, \verb|iftrue| (entra no bloco \verb|if|) e \verb|iffalse| (entra no \verb|else|).

O arquivo \verb|testes/add.rb|, por exemplo, contém um procedimento simples em IMP que executa algumas adições e imprime o resultado. O comando \verb|make add| executa esse teste e imprime o retorno.

\end{frame}
\end{document}

